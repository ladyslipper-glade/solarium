
\section{Computation}
\label{sec:computation}

Each object in the solar system has a location given by three coordinates. These coordinates
constitute the object's position vector. In addition each object has a velocity vector that
defines the object's motion. The position and velocity of an object are functions of time.

In our simulation time moves forward in discrete jumps called ``time steps.'' The object's
position and velocity are updated with each time step. The input of the computation consists of
three tables; one that gives the initial position of each object, another that gives the initial
velocity of each object, and a third that gives the mass of each object. The output of the
computation is a collection of tables with one table for each object giving the position of that
object for every time step of the simulation. Alternatively one could regard the output as a
single, large table with a row for each time step and a column for each object. Note that the
velocity of the objects are not output but velocities must be computed so that position can be
updated properly.

Real solar systems contain a large number of objects and have lifespans measured in billions of
years. A fully accurate simulation would thus create an immense amount of data. To make the
amount of data produced managable our simulation will track a much smaller number of objects and
use as large a time step as possible. Keeping the simulation accurate under these conditions is
a significant problem in numerical analysis. Although these matters are quite interesting they
are outside the scope of this exercise. We will build our software and then tinker with the
parameters to see how things behave.

In principle the physics involved in this simulation is very simple. Each object in our universe
exerts a gravitational attraction on every other object. To compute the motion of a object one
would follow the steps below.

\begin{enumerate}

\item Add up the graviational force on the object due to every other object in the universe. The
  result is a vector in three dimensions.

\item Since $F = ma$, the acceleration of the object can be computed by doing $F/m$ where $m$ is
  the mass of the object. The result is also a vector in three dimensions.

\item Since $a = dv/dt$, multiplying the acceleration vector by the length of a time step
  results in a vector giving the change in velocity over that time step. Add that change in
  velocity to the object's initial velocity vector to obtain its velocity vector at the end of
  the time step.

\item Since $v = dx/dt$, multiplying the velocity vector by the length of a time step results in
  a vector giving the change in position over that time step. Add that change in position to the
  object's initial position vector to obtain it's position vector at the end of the time step.

\item Output the object's new position vector. Repeat for all objects in the universe. Advance the
  master clock by one time step and repeat the entire computation.

\end{enumerate}

At the end of the computation we would have a table with one row for each time step and one
column for each object. In each table cell would be the position of a particular object at a
particular time. Scanning down a column would thus reveal the motion of a single object over
time. Scanning across a row would show the position of every object in the universe for a
particular time.

You may recognize this process as that of solving a differential equation using numerical
integration. In fact, the method I describe above for evaluating the integrals is quite
simplistic. Eventually we should replace this method with something more sophisticated. However,
if our primary goal is to explore parallel programming it is not critical to build a well
structured numerical computation.
