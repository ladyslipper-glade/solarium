
\section{Preliminaries}
\label{sec:preliminaries}

\subsection{Units and Scaling}

\textit{TODO: Finish me!}

\subsection{Input Data}

\textit{TODO: Finish me!}

\subsection{Visualization}

Once the results have been calculated, the next step is visualizing those results. A large table
of values is not easy for humans to interpret. Instead a program should be used that renders the
information in some easy to understand manner (typically graphically). This could be done
off-line and might even be best done by prebuilt tools. For example, MATLAB might be able to
help with the visualization. However, if necessary we can write our own program(s).

I'm imagining a program that displays each object in the universe as a dot on the screen. It
could then animate the image so that we could watch the objects moving in accordance with the
results of the calculation. Other forms of visualization might also be possible (chart the
motion of the center of mass perhaps?).
