
\section{Sequential}
\label{sec:sequential}

The first step in this project is to create a sequential (single threaded) program that solves
this problem. This program can be used as a performance baseline against which the parallel
programs can be compared. Ideally a parallel program with $n$ processing elements (threads,
cluster nodes, etc.) would run $n$ times faster than the sequential program.

The sequential program will be written in the C programming language. Many high performance
computations are written in Fortran, a language optimized for this kind of application. Fortran
programs can often outperform equivalent C programs by 10--20\%.

The sequential program is called \filename{Serial}. The program contains three parts.
\begin{enumerate}
\item Read initial inputs.
\item Do the computation (loop over all time steps, etc.).
\item Write the outputs.
\end{enumerate}

Because \filename{Serial} can produce a considerable amount of data for each time step, a
version of \filename{Serial} with practical limits on memory consumption will need to output
results as it goes along. This means that timing data must be gathered inside the program so
that the computation time can be clearly distinguished from the disk I/O time\footnote{Both
times are interesting, of course. Our parallel machine can do parallel disk access and so it
should be able to accelerate both computation and I/O operations.}.

% This paragraph should probably go into a separate section.
One interesting question is: can the program go fast enough to allow real time visualization of
the results? That is, can it compute results as fast as they are needed by the visualization
software? Most major simulations are not able to do this when run on realistic problem sizes.
However, if the simulation can run that quickly it creates the possibility of making the
simulation interactive.

\subsection{Input}
\label{sec:sequential-input}

Each problem instance is described by a small text file that defines the size of the problem and
specifies the location of the other input files. \filename{Serial} takes the name of this
problem instance file as a command line argument; it is the only input file that needs to be
directly specified.

The format of the problem instance file is as a sequence of name=value pairs. Blank lines are
ignored. Leading and trailing white space is ignored. White space around the `=' character is
ignored. However, embedded white space in names or values is significant. Comments can be
introduced with a `\#' character; all text from that character to the end of the line is
ignored. What follows is a description of the various problem instance settings that can be
stored in the problem instance file. Except as otherwise noted, the settings can appear in any
order in the file. All settings must be present; there are no defaults.

\begin{description}
\item[Version] (integer) The version of the problem instance file in use. This item must be the
  first non-blank, non-comment line in the file. Currently, only version 1 is supported.
\item[Units] (string) An identifier that indicates what system of units are being used by the
  simulation. The allows values of this identifier are not specified here. However,
  \filename{serial} will abort with an error if it does not recognize the identifier used.
\item[N] (integer) The number of objects in the simulation. This value is used to specify the
  size of the input tables. $N > 0$ is required.
\item[MassTable] (string) The name of the file containing the object mass data.
\item[PositionTable] (string) The name of the file containing the initial positions of all the
  objects.
\item[VelocityTable] (string) The name of the file containing the initial velocity of all the
  objects.
\end{description}

The first object (object \#0) is treated in a special way by the visualization system. It is
intended to represent the sun/star that dominates the system. However, there is no special
handling of this object by the computations. Multiple star systems, or systems without stars,
can also be readily simulated.

Here is a sample problem instance file
\begin{verbatim}
# This is a sample file.

Version = 1   # Only version allowed. Must be first.
N=100
Units = mks

# Use a common mass table.
MassTable = C:\Documents and Settings\Peter\serial\mass1.txt

PositionTable = run1-position.txt
VelocityTable = run1-velocity.txt
\end{verbatim}

The input tables use a format similar to the problem instance file. They are text files with one
record on each line. Blank lines and comments introduced with `\#' are ignored. The first line
must be a Version setting with a value that agrees with the Version setting in the problem
instance file. Currently, only Version 1 is supported. The next two lines must contain a unit
setting and a value of $N$ that agree with what is in the problem instance file. This redundancy
ensures that one does not accidentally create a problem instance with, for example, an
inconsistent system of units.

The number of records in a table file must, naturally, agree with the specified value of $N$. If
it does not \filename{Serial} will display an error message and abort. If the number of records
is not correct \filename{Serial} assumes there is an error in the file and won't spend time
doing a run on bad data.

Each data line starts with an integer $k$ in the range $1 \le k \le N$. The lines must be
numbered in order; it is a fatal error if they are not. Following the integer is a comma
separated list of values. The mass table contains a single value on each line. The other two
tables contain $(x, y, z)$ coordinates of the initial position or velocity as appropriate. The
format of each value follows the usual conventions for floating point numbers. Extra spaces
separating the values are ignored.

Here is a sample velocity table for $N=4$
\begin{verbatim}
# This is a sample.

Version = 1
N = 4
Units = xyz

# The actual data, one (V_x, V_y, V_z) record for each star.
1  1.23e-3, -4.24e-2,  5.04e+1
2  8.79e-4,  5.63e-1,  8.88e+0
3  2.11e-2,  2.43e+2, -6.55e+1
4 -3.22e-1,  4.34e+1,  6.93e+1
\end{verbatim}

\textit{TODO: Finish me!}
